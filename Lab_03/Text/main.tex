%!TEX TS-program = xelatex

% Шаблон документа LaTeX создан в 2018 году
% Алексеем Подчезерцевым
% В качестве исходных использованы шаблоны
% 	Данилом Фёдоровых (danil@fedorovykh.ru) 
%		https://www.writelatex.com/coursera/latex/5.2.2
%	LaTeX-шаблон для русской кандидатской диссертации и её автореферата.
%		https://github.com/AndreyAkinshin/Russian-Phd-LaTeX-Dissertation-Template

\documentclass[a4paper,14pt]{article}

\input{data/preambular.tex}
\begin{document} % конец преамбулы, начало документа
\input{data/title.tex}
%\tableofcontents
\pagebreak
\section{Инвертирующий интегратор}

В программе Micro-Cap была создана схема инвертирующего (рис. \ref{fig:schema_int_base}), а так же интегратора с шунтирующим сопротивлением (рис. \ref{fig:schema_int}).

\begin{figure}[H]
	\centering
	\includegraphics[width=\linewidth]{../imgs/schema_int_base}
	\caption{Схема инвертирующего интегратора}
	\label{fig:schema_int_base}
\end{figure}

\begin{figure}[H]
	\centering
	\includegraphics[width=\linewidth]{../imgs/schema_int}
	\caption{Схема инвертирующего интегратора с шунтирующим сопротивлением}
	\label{fig:schema_int}
\end{figure}

\subsection{Исследование выходных сигналов на различных частотах}

\subsubsection{Гармонический сигнал}

\begin{figure}[H]
	\centering
	\includegraphics[width=0.9\linewidth]{../imgs/tran_sin_10Hz}
	\caption{Осциллограмма для гармонического сигнала при частоте 10 Гц}
	\label{fig:tran_sin_10Hz}
\end{figure}

Амплитуда выходного сигнала $27.2V$ (рис. \ref{fig:tran_sin_10Hz})

\begin{figure}[H]
	\centering
	\includegraphics[width=0.7\linewidth]{../imgs/tran_sin_500Hz}
	\caption{Осциллограмма для гармонического сигнала при частоте 500 Гц}
	\label{fig:tran_sin_500Hz}
\end{figure}

Амплитуда выходного сигнала $0.639V$ (рис. \ref{fig:tran_sin_500Hz})

\subsubsection{Квадратичный сигнал}

\begin{figure}[H]
	\centering
	\includegraphics[width=0.7\linewidth]{../imgs/tran_square_10Hz}
	\caption{Осциллограмма для квадратичного сигнала при частоте 10 Гц}
	\label{fig:tran_square_10Hz}
\end{figure}

Амплитуда выходного сигнала $27.2V$ (рис. \ref{fig:tran_square_10Hz})

\begin{figure}[H]
	\centering
	\includegraphics[width=0.7\linewidth]{../imgs/tran_square_500Hz}
	\caption{Осциллограмма для квадратичного сигнала при частоте 500 Гц}
	\label{fig:tran_square_500Hz}
\end{figure}

Амплитуда выходного сигнала $2.25V$ (рис. \ref{fig:tran_square_500Hz})

\subsubsection{Треугольный сигнал}

\begin{figure}[H]
	\centering
	\includegraphics[width=0.7\linewidth]{../imgs/tran_triangle_10Hz}
	\caption{Осциллограмма для треугольного сигнала при частоте 10 Гц}
	\label{fig:tran_triangle_10Hz}
\end{figure}

Амплитуда выходного сигнала $27.5V$ (рис. \ref{fig:tran_triangle_10Hz})

\begin{figure}[H]
	\centering
	\includegraphics[width=0.7\linewidth]{../imgs/tran_triangle_500Hz}
	\caption{Осциллограмма для треугольного сигнала при частоте 500 Гц}
	\label{fig:tran_triangle_500Hz}
\end{figure}

Амплитуда выходного сигнала $1.25V$ (рис. \ref{fig:tran_triangle_500Hz})

\subsection{Исследование АЧХ и ФЧХ интегратора}

\begin{figure}[H]
	\centering
	\includegraphics[width=\linewidth]{../imgs/int_fr}
	\caption{АЧХ интегратора}
	\label{fig:int_fr}
\end{figure}


Частота среза: $10.6Hz$

Полоса пропускания: $1Hz - 10.6Hz$

\begin{figure}[H]
	\centering
	\includegraphics[width=\linewidth]{../imgs/int_pr}
	\caption{ФЧХ интегратора}
	\label{fig:int_pr}
\end{figure}

Максимальный фазовый сдвиг: $173.5^{\circ}$

Наклон ФЧХ: $-19.811$


\end{document} % конец документа






