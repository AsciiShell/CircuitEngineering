%!TEX TS-program = xelatex

% Шаблон документа LaTeX создан в 2018 году
% Алексеем Подчезерцевым
% В качестве исходных использованы шаблоны
% 	Данилом Фёдоровых (danil@fedorovykh.ru) 
%		https://www.writelatex.com/coursera/latex/5.2.2
%	LaTeX-шаблон для русской кандидатской диссертации и её автореферата.
%		https://github.com/AndreyAkinshin/Russian-Phd-LaTeX-Dissertation-Template

\documentclass[a4paper,14pt]{article}

\input{data/preambular.tex}
\begin{document} % конец преамбулы, начало документа
\input{data/title.tex}
%\tableofcontents
\pagebreak
\section{Исходные и рассчитанные параметры}

Выбор и подсчет исходных значений:
\begin{itemize}

\item Тип фильтра: Баттерворта ($\alpha = 1.414; \varepsilon = 1.000$)

\item $f_{cp} = 2kHz$

\item Выбор R

$R_1 = R_2 = R_a = R = 10 k\Omega$

\item Определение C

$f_{cp} = \dfrac{1}{2\pi RC}$

$C = \dfrac{1}{2\pi Rf_{cp}} = \dfrac{1}{2\pi * 10k\Omega * 2kHz}$

$C_1 = C_2 = C = 7.958 nF$

\item $R_b = (2-\alpha)R_a = (2 - 1.414)10k\Omega = 5860 \Omega$

\item $Ku_{teor} = \dfrac{U_{out}}{U_{in}} = \dfrac{I_{OC}(R_b + R_a)}{I_{OC}R_a} = \dfrac{R_b}{R_a} + 1 = \dfrac{(2 - \alpha) R_a}{R_a} + 1 = 3 - \alpha = 1.586$
\end{itemize}

\section{Фильтр низких частот}

\subsection{Построение схемы}

Схема фильтра нижних частот представлена на рис. \ref{fig:shfnch}.

\begin{figure}[H]
	\centering
	\includegraphics[width=0.85\linewidth]{../imgs/FNCH/sh_fnch}
	\caption{Схема фильтра низких частот}
	\label{fig:shfnch}
\end{figure}

\subsection{Получение АЧХ и ФЧХ}

Построим графики АЧХ и ФЧХ (рис. \ref{fig:afchn}).

\begin{figure}[H]
	\centering
	\includegraphics[width=0.85\linewidth]{../imgs/FNCH/AFcH_N}
	\caption{АЧХ и ФЧХ с измерением частоты среза}
	\label{fig:afchn}
\end{figure}

Фактическая частота среза получилась $1.999kHz \approx 2kHz$.

Полоса пропускания находится в интервале $(0;2)kHz$.

\subsection{Получение динамических характеристик}

Построим график динамической характеристики для частот $500Hz$ (рис. \ref{fig:garm500n}) и $50kHz$ (рис. \ref{fig:garm50kn}).

\begin{figure}[H]
	\centering
	\includegraphics[width=0.95\linewidth]{../imgs/FNCH/garm_500_N}
	\caption{Динамическая характеристика ФНЧ для частоты $500Hz$}
	\label{fig:garm500n}
\end{figure}

Размах напряжения входного сигнала $2V$

Размах напряжения выходного сигнала $3.172V$

Рассчитаем фактический коэффициент усиления:

$Ku_{fact} = \dfrac{3.172V}{2V} = 1.586$

$Ku_{teor} = 1.586$

$Ku_{teor} = Ku_{fact}$

Коэффициент передачи 

\begin{figure}[H]
	\centering
	\includegraphics[width=0.95\linewidth]{../imgs/FNCH/garm_50k_N}
	\caption{Динамическая характеристика ФНЧ для частоты $50kHz$}
	\label{fig:garm50kn}
\end{figure}

Напряжение на повторяющихся локальных максимумах $U_{out} \approx 5.5mV$

$Ku = \dfrac{5.5mV}{1V} = 0.0055$

Выходное напряжение пренебрежимо мало по отношению ко входному, из этого можно сделать вывод, что сигнал данной частоты не пропускается через данный фильтр.

\subsection{Исследование влияния погрешностей}

Найдем зависимость между частотой среза и погрешностью сопротивления (рис. \ref{fig:pogrrn}). 

$R_{pogr} = 0.9R$

\begin{figure}[H]
	\centering
	\includegraphics[width=0.85\linewidth]{../imgs/FNCH/pogr_R_N}
	\caption{АЧХ и ФЧХ для ФНЧ с уменьшенным сопротивлением}
	\label{fig:pogrrn}
\end{figure}

В данном случае частота среза стала больше, и составляет $2.22KHz$.

Найдем зависимость между частотой среза и погрешностью емкости (рис. \ref{fig:pogrcn}). 

$C_{pogr} = 0.9C$

\begin{figure}[H]
	\centering
	\includegraphics[width=0.85\linewidth]{../imgs/FNCH/pogr_C_N}
	\caption{АЧХ и ФЧХ для ФНЧ с уменьшенной емкостью}
	\label{fig:pogrcn}
\end{figure}

В данном случае частота среза стала больше, и также составляет $2.22KHz$.

Из проделанных опытов можно сделать вывод, что уменьшение сопротивления или емкости увеличивает частоту среза.

Действительно, частота среза обратно зависит $R$ и $C$ и находится по формуле:

$f_{cp} = \dfrac{1}{2\pi RC}$

Можно заметить, что при изменении $R$ и $C$ в одно и то же число раз, получена одна и та же частота среза.
Это можно также объяснить формулой, по которой находится частота среза.

\section{Фильтр высоких частот}

\subsection{Построение схемы}

Схема фильтра высоких частот представлена на рис. \ref{fig:shfvch}.


\begin{figure}[H]
	\centering
	\includegraphics[width=0.85\linewidth]{../imgs/FVCH/sh_fvch}
	\caption{Схема фильтра высоких частот}
	\label{fig:shfvch}
\end{figure}


\subsection{Получение АЧХ и ФЧХ}

Построим графики АЧХ и ФЧХ (рис. \ref{fig:afchv1}).

\begin{figure}[H]
	\centering
	\includegraphics[width=0.85\linewidth]{../imgs/FVCH/AFcH_V1}
	\caption{АЧХ и ФЧХ с измерением частоты среза}
	\label{fig:afchv1}
\end{figure}


Фактическая частота среза получилась $1.998kHz \approx 2kHz$.

Полоса пропускания находится в интервале $(2kHz; 2.148MHz)$.


\subsection{Получение динамических характеристик}

Построим график динамической характеристики для частот $50Hz$ (рис. \ref{fig:garm50v}) и $50kHz$ (рис. \ref{fig:garm50kv}).

\begin{figure}[H]
	\centering
	\includegraphics[width=0.95\linewidth]{../imgs/FVCH/garm_50_V}
	\caption{Динамическая характеристика ФВЧ для частоты $50Hz$}
	\label{fig:garm50v}
\end{figure}


Напряжение на повторяющихся локальных максимумах $U_{out} \approx 4mV$

$Ku = \dfrac{4mV}{1V} = 0.004$

Выходное напряжение пренебрежимо мало по отношению ко входному, из этого можно сделать вывод, что сигнал данной частоты не пропускается через данный фильтр.


\begin{figure}[H]
	\centering
	\includegraphics[width=0.95\linewidth]{../imgs/FVCH/garm_50k_V}
	\caption{Динамическая характеристика ФВЧ для частоты $50kHz$}
	\label{fig:garm50kv}
\end{figure}

Размах напряжения входного сигнала $2V$

Размах напряжения выходного сигнала $3.172V$

Рассчитаем фактический коэффициент усиления:

$Ku_{fact} = \dfrac{3.172V}{2V} = 1.586$

$Ku_{teor} = 1.586$

$Ku_{teor} = Ku_{fact}$


\subsection{Исследование влияния погрешностей}

Найдем зависимость между частотой среза и погрешностью сопротивления (рис. \ref{fig:pogrrv}). 

$R_{pogr} = 1.1R$

\begin{figure}[H]
	\centering
	\includegraphics[width=0.85\linewidth]{../imgs/FVCH/pogr_R_V}
	\caption{АЧХ и ФЧХ для ФВЧ с увеличенным сопротивлением}
	\label{fig:pogrrv}
\end{figure}

В данном случае частота среза стала меньше, и составляет $1.817KHz$.

Найдем зависимость между частотой среза и погрешностью емкости (рис. \ref{fig:pogrcv}). 

$C_{pogr} = 1.1C$

\begin{figure}[H]
	\centering
	\includegraphics[width=0.85\linewidth]{../imgs/FVCH/pogr_C_V}
	\caption{АЧХ и ФЧХ для ФВЧ с увеличенной емкостью}
	\label{fig:pogrcv}
\end{figure}

В данном случае частота среза стала меньше, и также составляет $1.817KHz$.

Из проделанных опытов можно сделать вывод, что увеличение сопротивления или емкости уменьшает частоту среза.

Действительно, частота среза обратно зависит $R$ и $C$ и находится по формуле:

$f_{cp} = \dfrac{1}{2\pi RC}$

Можно заметить, что при изменении $R$ и $C$ в одно и то же число раз, получена одна и та же частота среза.
Это можно также объяснить формулой, по которой находится частота среза.

\section{Подгон параметров}

Изменим схемы следующим образом:

\begin{figure}[H]
	\centering
	\includegraphics[width=0.85\linewidth]{../imgs/FNCH/sh_fnch_p}
	\caption{ФНЧ с подогнанными параметрами}
	\label{fig:shfnchp}
\end{figure}


\begin{figure}[H]
	\centering
	\includegraphics[width=0.85\linewidth]{../imgs/FVCH/sh_fvch_p}
	\caption{ФВЧ с подогнанными параметрами}
	\label{fig:shfvchp}
\end{figure}

Полученные АЧХ и ФЧХ для ФНЧ и ФВЧ:

\begin{figure}[H]
	\centering
	\includegraphics[width=0.85\linewidth]{../imgs/FNCH/gr_fnch_p1}
	\caption{АЧХ и ФЧХ для ФНЧ}
	\label{fig:grfnchp1}
\end{figure}


\begin{figure}[H]
	\centering
	\includegraphics[width=0.85\linewidth]{../imgs/FVCH/gr_fvch_p1}
	\caption{АЧХ и ФЧХ для ФВЧ}
	\label{fig:grfvchp1}
\end{figure}

$Ku_{teor}$ идеально совпадает с $Ku_{pract}$ для ФНЧ и ФВЧ и без подгона.

\end{document} % конец документа






