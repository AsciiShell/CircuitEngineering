%!TEX TS-program = xelatex

% Шаблон документа LaTeX создан в 2018 году
% Алексеем Подчезерцевым
% В качестве исходных использованы шаблоны
% 	Данилом Фёдоровых (danil@fedorovykh.ru) 
%		https://www.writelatex.com/coursera/latex/5.2.2
%	LaTeX-шаблон для русской кандидатской диссертации и её автореферата.
%		https://github.com/AndreyAkinshin/Russian-Phd-LaTeX-Dissertation-Template

\documentclass[a4paper,14pt]{article}

\input{data/preambular.tex}
\begin{document} % конец преамбулы, начало документа
\input{data/title.tex}
%\tableofcontents
\pagebreak
\section{Исходные и рассчитанные параметры}

Выбор и подсчет исходных значений:
\begin{itemize}

\item Тип фильтра: Баттерворта ($\alpha = 1.414; \varepsilon = 1.000$)

\item $f_{cp} = 2kHz$

\item Выбор R

$R_1 = R_2 = R_a = R = 10 k\Omega$

\item Определение C

$f_{cp} = \dfrac{1}{2\pi RC}$

$C = \dfrac{1}{2\pi Rf_{cp}} = \dfrac{1}{2\pi * 10k\Omega * 2kHz}$

$C_1 = C_2 = C = 7.958 nF$

\item $R_b = (2-\alpha)R_a = (2 - 1.414)10k\Omega = 5860 \Omega$

\item $Ku_{teor} = \dfrac{U_{out}}{U_{in}} = \dfrac{I_{OC}(R_b + R_a)}{I_{OC}R_a} = \dfrac{R_b}{R_a} + 1 = \dfrac{(2 - \alpha) R_a}{R_a} + 1 = 3 - \alpha = 1.586$
\end{itemize}

\section{Фильтр низких частот}

\subsection{Построение схемы}

Схема фильра нижних частот представлена на рис. \ref{fig:shfnch}.

\begin{figure}[H]
	\centering
	\includegraphics[width=0.85\linewidth]{../imgs/FNCH/sh_fnch}
	\caption{Схема фильтра низких частот}
	\label{fig:shfnch}
\end{figure}

\subsection{Получение АЧХ и ФЧХ}

Построим графики АЧХ и ФЧХ (рис. \ref{fig:afchn}).

\begin{figure}[H]
	\centering
	\includegraphics[width=0.85\linewidth]{../imgs/FNCH/AFcH_N}
	\caption{АЧХ и ФЧХ с измерением частота среза}
	\label{fig:afchn}
\end{figure}

Фактическая частота среза получилась $1.999kHz \approx 2kHz$.

Полоса пропускания находится в интервале $(0;2)kHz$.

\subsection{Получение динамических характеристик}

Построим график динамической характеристики для частот $500Hz$ (рис. \ref{fig:garm500n}) и $50kHz$ (рис. \ref{fig:garm50kn}).

\begin{figure}[H]
	\centering
	\includegraphics[width=0.95\linewidth]{../imgs/FNCH/garm_500_N}
	\caption{Динамическая характеристика ФНЧ для частоты $500Hz$}
	\label{fig:garm500n}
\end{figure}

Размах напряжения входного сигнала $2V$

Размах напряжения выходного сигнала $3.13V$

Рассчитаем фактический коэффициент усиления:

$Ku_{fact} = \dfrac{3.13V}{2V} = 1.565$

$Ku_{teor} = 1.586$

$Ku_{teor} \approx Ku_{fact}$

\begin{figure}[H]
	\centering
	\includegraphics[width=0.95\linewidth]{../imgs/FNCH/garm_50k_N}
	\caption{Динамическая характеристика ФНЧ для частоты $50kHz$}
	\label{fig:garm50kn}
\end{figure}

Напряжение на повторяющихся локальных максимумах $U_{out} \approx 5.5mV$

$Ku = \dfrac{5.5mV}{1V} = 0.0055$

Выходное напряжение пренебрежимо мало по отношению ко входному, из этого можно сделать вывод, что сигнал данной частоты не пропускается через данный фильтр.



\end{document} % конец документа






