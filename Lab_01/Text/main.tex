%!TEX TS-program = xelatex

% Шаблон документа LaTeX создан в 2018 году
% Алексеем Подчезерцевым
% В качестве исходных использованы шаблоны
% 	Данилом Фёдоровых (danil@fedorovykh.ru) 
%		https://www.writelatex.com/coursera/latex/5.2.2
%	LaTeX-шаблон для русской кандидатской диссертации и её автореферата.
%		https://github.com/AndreyAkinshin/Russian-Phd-LaTeX-Dissertation-Template

\documentclass[a4paper,14pt]{article}

\input{data/preambular.tex}
\begin{document} % конец преамбулы, начало документа
\input{data/title.tex}
%\tableofcontents
\pagebreak
\section{Исследование ключа на биполярном транзисторе}

В программе Micro-Cap была создана схема ключа на биполярном транзисторе с общим эмиттером (рис. \ref{fig:btsh}).

\begin{figure}[H]
	\centering
	\includegraphics[width=0.7\linewidth]{image/BT_sh}
	\caption{Ключ на биполярном транзисторе с общим эмиттером}
	\label{fig:btsh}
\end{figure}

Далее получен график переходных процессов (рис. \ref{fig:btgrafper}).

\begin{figure}[H]
	\centering
	\includegraphics[width=0.7\linewidth]{image/BT_graf_per}
	\caption{График переходных процессов для БТ транзистора}
	\label{fig:btgrafper}
\end{figure}

С помощью инструмента «Вертикальная размерная линия» получаем $\Delta U$ (рис. \ref{fig:btgrafper1}).

$$\Delta U = 4.749 V$$

\begin{figure}[H]
	\centering
	\includegraphics[width=0.2\linewidth]{image/BT_graf_per1}
	\caption{Применение инструмента «Вертикальная размерная линия» для измерения $\Delta U$}
	\label{fig:btgrafper1}
\end{figure}

На полученном графике переходных процессов определяем длительность фронтов $t^{01}_\phi$ и $t^{10}_\phi$.
Для этого необходимо рассчитать начало и конец нужного фронта.

\begin{itemize}
	\item Для переднего фронта: 
	\begin{itemize}
		\item Начало: $U^0 + \Delta * 10\%$
		\item Конец:  $U^1 - \Delta * 10\%$
	\end{itemize}
	
	\item Для заднего фронта: 
	\begin{itemize}
		\item Начало: $U^1 - \Delta * 10\%$
		\item Конец:  $U^0 + \Delta * 10\%$
	\end{itemize}
\end{itemize}

\begin{figure}[H]
	\centering
	\includegraphics[width=0.7\linewidth]{image/BT_graf_t01}
	\caption{Измерение переднего фронта $t^{01}_\phi$}
	\label{fig:btgraft01}
\end{figure}

\begin{figure}[H]
	\centering
	\includegraphics[width=0.7\linewidth]{image/BT_graf_t10}
	\caption{Измерение заднего фронта $t^{10}_\phi$}
	\label{fig:btgraft10}
\end{figure}

Из рисунков \ref{fig:btgraft01} и \ref{fig:btgraft10} получаем:

$$t^{01}_\phi = 227.88$$
$$t^{01}_\phi = 145.242$$

\end{document} % конец документа