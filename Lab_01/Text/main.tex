%!TEX TS-program = xelatex

% Шаблон документа LaTeX создан в 2018 году
% Алексеем Подчезерцевым
% В качестве исходных использованы шаблоны
% 	Данилом Фёдоровых (danil@fedorovykh.ru) 
%		https://www.writelatex.com/coursera/latex/5.2.2
%	LaTeX-шаблон для русской кандидатской диссертации и её автореферата.
%		https://github.com/AndreyAkinshin/Russian-Phd-LaTeX-Dissertation-Template

\documentclass[a4paper,14pt]{article}

\input{data/preambular.tex}
\begin{document} % конец преамбулы, начало документа
\input{data/title.tex}
%\tableofcontents
\pagebreak
\section{Исследование ключа на биполярном транзисторе}
\subsection{Исследование ключа на биполярном транзисторе без нагрузки}

В программе Micro-Cap была создана схема ключа на биполярном транзисторе с общим эмиттером (рис. \ref{fig:btsh}).

\begin{figure}[H]
	\centering
	\includegraphics[width=0.7\linewidth]{image/BT_sh}
	\caption{Схема ключ на биполярном транзисторе с общим эмиттером без нагрузки}
	\label{fig:btsh}
\end{figure}

Далее получен график переходных процессов (рис. \ref{fig:btgrafper}).

\begin{figure}[H]
	\centering
	\includegraphics[width=0.7\linewidth]{image/BT_graf_per}
	\caption{График переходных процессов для БТ транзистора без нагрузки}
	\label{fig:btgrafper}
\end{figure}

С помощью инструмента «Вертикальная размерная линия» получаем $\Delta U$ (рис. \ref{fig:btgrafper1}).

$$\Delta U = 4.749 V$$

\begin{figure}[H]
	\centering
	\includegraphics[width=0.2\linewidth]{image/BT_graf_per1}
	\caption{Применение инструмента «Вертикальная размерная линия» для измерения $\Delta U$}
	\label{fig:btgrafper1}
\end{figure}

На полученном графике переходных процессов определяем длительность фронтов $t^{01}_\phi$ и $t^{10}_\phi$.
Для этого необходимо рассчитать начало и конец нужного фронта.

\begin{itemize}
	\item Для переднего фронта: 
	\begin{itemize}
		\item Начало: $U^0 + \Delta * 10\%$
		\item Конец:  $U^1 - \Delta * 10\%$
	\end{itemize}
	
	\item Для заднего фронта: 
	\begin{itemize}
		\item Начало: $U^1 - \Delta * 10\%$
		\item Конец:  $U^0 + \Delta * 10\%$
	\end{itemize}
\end{itemize}

\begin{figure}[H]
	\centering
	\includegraphics[width=0.7\linewidth]{image/BT_graf_t01}
	\caption{Измерение переднего фронта $t^{01}_\phi$}
	\label{fig:btgraft01}
\end{figure}

\begin{figure}[H]
	\centering
	\includegraphics[width=0.7\linewidth]{image/BT_graf_t10}
	\caption{Измерение заднего фронта $t^{10}_\phi$}
	\label{fig:btgraft10}
\end{figure}

Из рисунков \ref{fig:btgraft01} и \ref{fig:btgraft10} получаем:

$$t^{01}_\phi = 227.88$$
$$t^{01}_\phi = 145.242$$

Далее измерим задержки переключения для $R_g = 5KOm$ и $R_g = 1KOm$, соответствующие графики представлены на рис. \ref{fig:btgrafzad5} и рис. \ref{fig:btgrafzad1}.

\begin{figure}[H]
	\centering
	\includegraphics[width=0.4\linewidth]{image/BT_graf_zad_5}
	\caption{Значения $t^{01}_{zad}$ и $t^{10}_{zad}$ при $R_g = 5KOm$}
	\label{fig:btgrafzad5}
\end{figure}


\begin{figure}[H]
	\centering
	\includegraphics[width=0.4\linewidth]{image/BT_graf_zad_1}
	\caption{Значения $t^{01}_{zad}$ и $t^{10}_{zad}$ при $R_g = 1KOm$}
	\label{fig:btgrafzad1}
\end{figure}

Из графиков (рис. \ref{fig:btgrafzad5}, рис. \ref{fig:btgrafzad1}) получаем:

При $R_g = 5KOm$: $t^{01}_{zad} = 235.392ns$ и $t^{10}_{zad} = 107.679ns$

При $R_g = 1KOm$: $t^{01}_{zad} = 523.372ns$ и $t^{10}_{zad} = 21.963ns$

\textbf{Вывод:} при уменьшении сопротивления цепи базы $t^{01}_{zad}$ возрастает с $235ns$ до $523ns$, при этом  $t^{10}_{zad}$ наоборот падает со $108ns$ до $22ns$ 


Рассмотрим график анализа по постоянному току DC (рис. \ref{fig:btdc}), отметим на нем 2 точки и по их координатам посчитаем коэффициент передачи.

\begin{figure}[H]
	\centering
	\includegraphics[width=0.4\linewidth]{image/BT_DC}
	\caption{График анализа по постоянному току DC}
	\label{fig:btdc}
\end{figure}

Из формулы прямой $\dfrac{x-x_1}{x_2 - x_1} = \dfrac{y-y_1}{y_2 - y_1}$ получаем:

%$$\dfrac{x-729.429}{1054 - 729.429} = \dfrac{y-3.720}{1.584 - 3.720}$$

\begin{equation}
y = x * \dfrac{y_2 - y_1}{x_2 - x_1} - x_1 * \dfrac{y_2 - y_1}{x_2 - x_1} + y_1
\label{eq:line_bez_nagr}
\end{equation}

В формуле \ref{eq:line_bez_nagr} нас интересует коэффициент перед $x$, его абсолютное значение и будет коэффициентом $K$: 

\begin{equation}
K = \left|\dfrac{y_2 - y_1}{x_2 - x_1}\right|
\label{eq:K_bez_nagr}
\end{equation}

Подставим координаты точек из рис. \ref{fig:btdc} в формулу \ref{eq:K_bez_nagr}:

\begin{equation}
K = \left|\dfrac{1.584 - 3.720}{1054 - 729.429} = \right|
\label{eq:K_bez_nagr_p}
\end{equation}

TODO:!!!!!!!!!!!!!!!!!!!!!!!!!!!!!! !!!!!!!!!!!!!!!!!!!!!!!!!!!!!!!!!! !!!!!!!!!!!!!!!!!!!!!!!!!!!!!!!!!!!! !!!!!!!!!!!!!!!!!!!!!!!!!!!!!!!!!!!!!!!!!!!!!!!!!!!!!!!!!!!!!!!!!!!!!!!!!!!!!!!!!!!!

\subsection{Исследование ключа на биполярном транзисторе с нагрузкой}

Для исследования влияния на динамические параметры и формирующие свойства ключа параметров нагрузки добавим к схеме ключа на биполярном транзисторе нагрузку рис. \ref{fig:btshnagr}.

\begin{figure}[H]
	\centering
	\includegraphics[width=0.7\linewidth]{image/BT_sh_nagr}
	\caption{Схема ключ на биполярном транзисторе с общим эмиттером с нагрузкой}
	\label{fig:btshnagr}
\end{figure}

\begin{figure}[H]
	\centering
	\includegraphics[width=0.7\linewidth]{image/BT_graf_nagr}
	\caption{График передаточной характеристики для ключа на биполярном транзисторе с нагрузкой}
	\label{fig:btgrafnagr}
\end{figure}

\begin{figure}[H]
	\centering
	\includegraphics[width=0.4\linewidth]{image/BT_graf_nagr_fr}
	\caption{Измерение фронтов $t^{01}_{\phi}$ и $t^{10}_{\phi}$}
	\label{fig:btgrafnagrfr}
\end{figure}

\begin{figure}[H]
	\centering
	\includegraphics[width=0.4\linewidth]{image/BT_graf_nagr_zad}
	\caption{Измерение задержки переключения $t^{01}_{zad}$ и $t^{10}_{zad}$}
	\label{fig:btgrafnagrzad}
\end{figure}


На рисунках \ref{fig:btgrafnagrfr} и \ref{fig:btgrafnagrzad} представлены измерения фронтов и времени переключения соответственно для ключа на биполярном транзисторе с нагрузкой. 

Получаем следующие значения:

$t^{01}_{\phi} = 211.525ns$

$t^{10}_{\phi} = 278.644ns$

$t^{01}_{zad} = 119.898ns$

$t^{10}_{zad} = 170.847ns$

\begin{table}[H]
	\caption{Сравнение динамических характеристик ключа на биполярном транзисторе с нагрузкой и без}
	\centering
	\begin{tabular}{|l|c|c|c|c|}
		\hline
		& $t^{01}_{\phi}, ns$ & $t^{10}_{\phi}, ns$ & $t^{01}_{zad}, ns$ & $t^{10}_{zad}, ns$ \\ \hline
		без нагрузки & 227 & 145 & 235 & 107 \\ \hline
		с нагрузкой & 211 & 278 & 119 & 170 \\ \hline
	\end{tabular}

\label{tab:sr_har}
\end{table}


\textbf{Вывод: } появление нагрузки уменьшает динамические характеристики при включении ($t^{01}_{\phi}$ и $t^{01}_{zad}$) и увеличивает при выключении ($t^{10}_{\phi}$ и $t^{10}_{zad}$) ключа.



\section{Исследование ключа на КМДП транзисторе}
\subsection{Исследование ключа на КМДП транзисторе без нагрузки}

\subsection{Исследование ключа на КМДП транзисторе с нагрузкой}



 \end{document} % конец документа