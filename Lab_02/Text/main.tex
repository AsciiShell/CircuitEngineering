%!TEX TS-program = xelatex

% Шаблон документа LaTeX создан в 2018 году
% Алексеем Подчезерцевым
% В качестве исходных использованы шаблоны
% 	Данилом Фёдоровых (danil@fedorovykh.ru) 
%		https://www.writelatex.com/coursera/latex/5.2.2
%	LaTeX-шаблон для русской кандидатской диссертации и её автореферата.
%		https://github.com/AndreyAkinshin/Russian-Phd-LaTeX-Dissertation-Template

\documentclass[a4paper,14pt]{article}

\input{data/preambular.tex}
\begin{document} % конец преамбулы, начало документа
\input{data/title.tex}
%\tableofcontents
\pagebreak
\section{Исследование ключа на биполярном транзисторе}

Была создана схема операционного усилителя (рис. \ref{fig:shop}).

\begin{figure}[H]
	\centering
	\includegraphics[width=0.9\linewidth]{image/sh_op}
	\caption{Собранная схема}
	\label{fig:shop}
\end{figure}

Построим осциллограммы для входного и выходного сигнала с гармонической, треугольной и прямоугольной формами сигнала с частотой 100кГц.

\begin{figure}[H]
	\centering
	\includegraphics[width=0.6\linewidth]{image/graf_1k_garm}
	\caption{Осциллограмма синусоидального сигнала}
	\label{fig:graf1kgarm}
\end{figure}

\begin{figure}[H]
	\centering
	\includegraphics[width=0.6\linewidth]{image/graf_1k_tre}
	\caption{Осциллограмма треугольного сигнала}
	\label{fig:graf1ktre}
\end{figure}

\begin{figure}[H]
	\centering
	\includegraphics[width=0.6\linewidth]{image/graf_1k_rect}
	\caption{Осциллограмма прямоугольного сигнала}
	\label{fig:graf1krect}
\end{figure}

Из осциллограмм получаем, что при частоте 100кГц сигнал инвертируется, но форма сигнала не меняется.

Измерим максимальную скорость нарастания выходного прямоугольного сигнала (рис. \ref{fig:makssk}).

\begin{figure}[H]
	\centering
	\includegraphics[width=0.5\linewidth]{image/maks_sk}
	\caption{Измерение максимальной скорости нарастания выходного сигнала}
	\label{fig:makssk}
\end{figure}

$$V = \dfrac{\Delta U}{\Delta t} = \dfrac{3.546B}{0.752212mc} = 4.714$$

Максимальная скорость нарастания равна $4.714\dfrac{B}{mc}$.

Построим АЧХ и ФЧХ.

\begin{figure}[H]
	\centering
	\includegraphics[width=0.7\linewidth]{image/ACHFH}
	\caption{АЧХ и ФЧХ}
	\label{fig:achfh}
\end{figure}


Найдем полосу пропускания на АЧХ (рис. \ref{fig:achh}). Для этого отступим от 3дБ максимального значения. 

\begin{figure}[H]
	\centering
	\includegraphics[width=0.7\linewidth]{image/ACHH}
	\caption{АЧХ с измеренной полосой пропускания}
	\label{fig:achh}
\end{figure}

Получаем полосу пропускания, равную 1.554МГц.

Получим суммарный сдвиг фаз до полосы пропускания(рис. \ref{fig:afh}).

\begin{figure}[H]
	\centering
	\includegraphics[width=0.7\linewidth]{image/AFH}
	\caption{Суммарный сдвиг фаз в полосе пропускания на ФЧХ}
	\label{fig:afh}
\end{figure}

Суммарный сдвиг фаз получился $66.983^{\circ}$.

Найдем частоту

 \end{document} % конец документа






